\documentclass{clv3}

\usepackage{hyperref}
\usepackage{xcolor}
\definecolor{darkblue}{rgb}{0, 0, 0.5}
\hypersetup{colorlinks=true,citecolor=darkblue, linkcolor=darkblue, urlcolor=darkblue}

\bibliographystyle{compling}

% test compatibility with algorithmic.sty
%\usepackage{algorithmic}

\issue{1}{1}{2016}

%Document Head
\dochead{CLV3 Class File Manual}

\runningtitle{How to Use CLV3 \LaTeX\ Class File}

\runningauthor{Odi\'e N. Gementera}

\begin{document}

\title{How to Use CLV3 \LaTeX\ Class File}

\author{Odi\'e N. Gementera\thanks{PITC Building, Pascor Drive, Sto. Ni\~no, Para\~naque City, 1700 Philippines. E-mail: o.gementera@spi-bpo.com.}}
\affil{Publishing / SPi}

\author{Another Author\thanks{PITC Building}}
\affil{Publishing / SPi}

\author{And Another Author}
\affil{Publishing / SPi}

\author{And Yet Another}
\affil{Publishing / SPi}

\maketitle

\begin{abstract}
This article describes how to use the ``CLV3'' class file, developed by {\sf SPi}, 
to produce typeset papers based on Computer Linguistics typesetting specifications 
for submission to MIT. This document was produced using ``{\tt clv3}'' class file 
and will be modified whenever there is an update in the layout specifications.
\end{abstract}

\section{Introduction}

This document applies to version 3 of CL class file. Prior style files such as 
``{\tt cl.sty}'' and ``{\tt coli.sty}'' do not have all of the features 
described here. It is assumed that the user has a basic knowledge of \LaTeX\ 
typesetting commands.

\section{Class File Options}

There are several options that can be used to switch the mode of MIT2 from normal 
article to manuscript style, or to different layout styles. This is specified in 
the usual \LaTeX\ way by declaring:

\verb|\documentclass[bookreview,manuscript]{clv3}|

%\subsection{Default Option}

\begin{deflist}
\item[bookreview] Sets the article layout for Book Review.
\item[brief] Sets the article layout for Briefly Noted.
\item[discussion] Sets the article layout for Squibs and Discussions.
\item[pubrec] Sets the article layout for Publication Received.
\item[shortpaper] Sets the article layout for Short Paper.
\item[manuscript] Sets the baseline spacing to double space. This 
option can be used in combination with other options.
\end{deflist}

By not declaring any option in the \verb|\documentclass| command the class file
will automatically set to standard article layout.

\section{Title Page}

The title page is created using the standard \LaTeX\ command \verb|\maketitle|. 
Before this command is declared, the author must declare all the data which are 
to appear in the title area.\footnote{$\backslash$maketitle is the command to execute all the title page information.}

\subsection{Volume, Number and Year}

The command \verb|\issue{vv}{nn}{yyyy}| is used in declaring the volume, number
and year of the article. The first argument is for the volume, the second argument 
is for the issue number. Volume and Issue number will appear on the even page 
running head opposite the journal name. The third argument is for the Year which 
will appear in the copyright line at the bottom of the title page.

\subsection{Document Head}

Document head is produced with the command \verb|\dochead{Document Head}|. Doc head 
will output differently, or may not appear at all, depending on the option used in the 
documentclass.

\subsection{Paper Title}

The paper title is declared like: \verb|\title{Computer Linguistic Article}|
in the usual \LaTeX manner. Line breaks may be inserted with (\verb|\\|) to equalize 
the length of the title lines.

\subsection{Authors}
The name and related information for authors is declared with the \verb|\author{}| command. 

The \verb|\thanks{}| command produces the ``first footnotes.''. \LaTeX\ \verb|\thanks| 
cannot accommodate multiple paragraphs, author will have to use a separate \verb|\thanks|
for each paragraph.

The \verb|\affil{}| command produces the author affiliations that appears right under 
the author's name.

\subsection{Running Headers}
The running heads are declared with the \verb|\runningtitle{Running Title}| for the 
journal name and \verb|\runningauthor{Author's Surname}| for author. These information 
will appear on the odd pages. For {\tt bookreview} option, odd page running head is 
automatically set to "Book Reviews". Even page running head is default to Computational 
Linguistics opposite volume and issue number.

\subsection{History Dates}

History dates are declared with \verb|\historydates{Submission received:...}|. This data
should contain Submission, Revised and Accepted date of the article. History dates appear
at the footnote area of title age.


\section{Abstract}

Abstract is the first part of a paper after \verb|\maketitle|. Abstract text is 
placed within the abstract environment:

\begin{verbatim}
\begin{abstract}
This is the abstract text . . .
\end{abstract}
\end{verbatim}

\section{Section Headings}

Section headings are declared in the usual \LaTeX\ way via \verb|\section{}|, 
\verb|\subsection{}|, \verb|\subsubsection{}|, and \verb|\paragraph{}|. The 
first 3 levels of section head will have Arabic numbering separated 
by period. The \verb|\paragraph{}| section will have the title head in Italics 
and at the same line with the first line of succeeding paragraph.

\section{Citations}
Citations in parentheses are declared using the \verb|\cite{}|
command, and appear in the text as follows: 
This technique is widely used \cite{woods}.
The command \verb|\citep{}| (cite parenthetical) is a synonym of \verb|\cite{}|.

Citations used in the sentence are declared using the \verb|\namecite{}|
commands, and appear in the text as follows: 
\namecite{woods} first described this technique.
The command \verb|\citet{}| (cite textual) is a synonym of \verb|\namecite{}|.

This style file is designed to be used with the BibTeX
style file \verb|compling.bst|.  Include the command
\verb|\bibliographystyle{compling}| in your source file.

Citation commands are based on the \verb|natbib| package;
for details on options and further variants of the commands,
see the \verb|natbib| documentation.  In particular, options 
exist to add extra text and page numbers.  For example, 
\verb|\cite[cf.][ch.\ 1]{winograd}| yields: \cite[cf.][ch.\ 1]{winograd}.

The following examples illustrate how citations appear both in the text
and in the references section at the end of this document.
\begin{enumerate}
\item Article in journal: 
 \namecite{akmajian};
 \namecite{woods}.
\item Book: 
  \namecite{altenberg};
  \namecite{winograd}.
\item Article in edited collection/Chapter in book: 
  \namecite{cutler};
  \namecite{sgall};
  \namecite{jurafsky}.
\item Technical report:
  \namecite{appelt};
  \namecite{robinson}.
\item Thesis or dissertation: 
  \namecite{baart};
  \namecite{spaerckjones};
  \namecite{cahn}.
\item Unpublished item: 
  \namecite{ayers}.
\item Conference proceedings: 
  \namecite{benoit}.
\item Paper published in conference proceedings: 
  \namecite{krahmer};
  \namecite{Copestake2001}.
\end{enumerate}


\section{Definition with Head}

Definition with head is declared by using the environment:
\\
\begin{verbatim}
\begin{definition}
Definition text. . .
\end{definition}
\end{verbatim}

This environment will generate the word {\bf ``Definition 1''} in bold on separate
line. The sequence number is generated for every definition environment. Definition 
data will have no indention on the first line while succeeding lines will have hang 
indention.

\section{Lists}

The usual \LaTeX\ itemize, enumerate and definition list environments are used 
in CLV3 style. 

To produce Numbered List use the environment:

\begin{verbatim}
\begin{enumerate}
\item First numbered list item
\item Second numbered list item
\item Third numbered list item
\end{enumerate}
\end{verbatim}

To produce Bulleted List use the environment:

\begin{verbatim}
\begin{itemize}
\item First bulleted list item
\item Second bulleted list item
\item Third bulleted list item
\end{itemize}
\end{verbatim}

To produce Definition List use the environment:

\begin{verbatim}
\begin{deflist}
\item[First]  Definition list item. . .
\item[Second] Definition list item. . .
\item[Third]  Definition list item. . .
\end{deflist}
\end{verbatim}

Additional list environment were also defined such as Unnumbered, Arabic and Alpha lists.

Unnumbered List is the list where item labels are not generated. To produce Unnumbered List use the environment:

\begin{verbatim}
\begin{unenumerate}
\item First list item
\item Second list item
\item Third list item
\end{unenumerate}
\end{verbatim}

To produce Arabic List use the environment:

\begin{verbatim}
\begin{arabiclist}
\item First arabic list item
\item Second arabic list item
\item Third arabic list item
\end{arabiclist}
\end{verbatim}

To produce Alpha List use the environment:

\begin{verbatim}
\begin{alphalist}
\item First alpha list item
\item Second alpha list item
\item Third alpha list item
\end{alphalist}
\end{verbatim}

All the list environments mentioned above can be nested with each other.

\subsection{Other List Types}

\subsubsection{Outline List or Example List}

\begin{verbatim}
\begin{exlist}
\item First outline list item. . .
\item Second outline list item. . .
\item Third outline list item. . .
\end{exlist}
\end{verbatim}

\subsubsection{Output Formula or Algorithm}

\begin{verbatim}
\begin{algorithm}
\item[Step 1] First item. . . 
\item[Step 2] Second item. . .
\end{algorithm}
\end{verbatim}

% test compatibility with algorithmic.sty
%\begin{algorithmic}
%\STATE i
%\end{algorithmic}

See sample on the {\tt COLI-template.pdf}.

\section{Word Formula or Displayed Text}

Word formula and displayed text are text that should be displayed in a 
separate line without indention. This are achieved by using the environment:

\begin{verbatim}
\begin{displaytext}
This is a sample of displayed text . . .
\end{displaytext}
\end{verbatim}

\section{Dialogue}

Dialogue text are presentation of people's conversation. These will be presented 
on a separate line where each dialogue starts with the name of speaker, followed by 
colon. Succeeding lines will be hang indented. To produce Dialogue use the environment:
\\
\begin{verbatim}
\begin{dialogue}
Speaker 1: dialogue. . .

Speaker 2: dialogue. . .
\end{dialogue}
\end{verbatim}


\noindent Please make sure to insert an empty line between dialogues.

\section{Extracts}

Extract text acts like quote, where left and right margins are indented.
To produce Extract use the environment:

\begin{verbatim}
\begin{extract}
This is an example of Extract text. . .
\end{extract}
\end{verbatim}

\noindent See sample on the {\tt COLI-template.pdf}.

\section{Theorem-like Environments}

There are several theorem-like environments defined in CLV3 class file. Theorem-like
environments generate the name of the theorem as label, and counter number in bold.

\subsection{Example}

To produce Example use the environment:

\begin{verbatim}
\begin{example}
This is Example text. . .
\end{example}
\end{verbatim}

\subsection{Lemma}

To produce Lemma use the environment:

\begin{verbatim}
\begin{lemma}
Lemma text. . .
\end{lemma}
\end{verbatim}

This produces the following output:
\begin{lemma}\label{lem}
Lemma text.
\end{lemma}
A small vertical space separates the end of the lemma
from the following text.

\subsection{Theorem}

To produce Theorem use the environment:

\begin{verbatim}
\begin{theorem}
Theorem text. . .
\end{theorem}
\end{verbatim}

This produces the following output:
\begin{theorem}\label{thm}
Theorem text.
\end{theorem}
\noindent
A small vertical space separates the end of the theorem
from the following text.

\subsection{Proof}

The proof environment produces a square at the end of the text. To produce Proof 
use the environment:

\begin{verbatim}
\begin{proof}
Proof text. . .
\end{proof}
\end{verbatim}

This produces the following output:
\begin{proof}\label{proof}
Proof text.
\end{proof}
A small vertical space separates the end of the lemma
from the following text.

\subsection{Unnumbered Theorem-like Environments}

There are also unnumbered version of some of the theorem-like environments. 
These are declared by using its asterisked version. Here are the three 
unnumbered version of theorem-like environments:

\begin{verbatim}
\begin{theorem*}
Unnumbered theorem text. . .
\end{theorem*}
\end{verbatim}

\section{Appendix}

Appendix is declared by issuing the command \verb|\appendix|. This will set 
the necessary labels to appendix's rule (i.e. (A.1) for equation number).

Sections inside Appendix are declared using \verb|\appendixsection{}|, which 
will produce {\bf Appendix A: Section Title} for first section.

Equation numbers are automatically set to (A.1), (A.2) and (A.3). Where the letters 
follow the current level of Appendix section. So equations on {\bf Appendix B} 
will have equation numbers as follow: (B.1), (B.2) and (B.3).

\section{Acknowledgments}

Acknowledgments are produce by using the environment:
\\
\begin{verbatim}
\begin{acknowledgments}
Acknowledgments text. . .
\end{acknowledgments}
\end{verbatim}

\section{Others}

Other items such as Equations, Figures, Tables and References are produced in
the standard \LaTeX\ typesetting.

\starttwocolumn
\bibliography{compling_style}

\end{document}
